\mnsection{Zadanie własne}

% Definicja zadania własnego.
\entry
Zadanie własne $A \in \mathbb{R}^{N \times N}$:
$(\lambda,x) \in \mathbb{C} \times \mathbb{C}^N$:
$Ax = \lambda x$;

% Fakt o wartościach własnych (1).
\entry
Gdy $\lambda$ dla $A$, to $(\lambda - \mu)$ dla $A-\mu I$;
% Fakt o wartościach własnych (2).
\entry
Gdy $\lambda$ dla $A$ nieos., to $\frac{1}{\lambda}$ dla $A^{-1}$;

% Metoda potęgowa wyznaczania wektora własnego.
% (9 XI/4.5/43)
\entry
\textbf{M}. \textbf{pot}.:
$\abs{\lambda_1} \!>\! \abs{\lambda_2} \!\geq\! \ldots \!\geq\! \abs{\lambda_n}
\!\!\Rightarrow\!
\fromloop{k}\{x_{k+1}=Ax_k; \normifyfrac[\infty][2]{x_{k+1}}\}$;

% Wniosek z powyższych faktów dla metody potęgowej.
\entry
M. pot. zastosowana do
$(A-\mu I)^{-1}$
będzie zbieżna do
$\lambda_i$,
czyli w. wł. najbliższej $\mu$.
Rzeczywiście, w. wł.
$(A-\mu I)^{-1}$
to
$\frac{1}{\lambda_i-\mu}$
i jeśli
$\frac{1}{\shortabs{\lambda_i-\mu}} > \frac{1}{\shortabs{\lambda_j-\mu}}, j\neq i$,
to
$\frac{1}{\lambda_i-\mu}$
jest dominująca;

% Odwrotna metoda potęgowa.
% (MP: 16 XI/1.5/2)
\entry
\textbf{Odw. m. pot.}:
$\text{for } k=0,\ldots$:
$x_{k+1} = (A-\mu I)^{-1}x_k$;
$x_{k+1} = x_{k+1} / \norm{x_{k+1}}$;
$\bigoh{n^3}$;
% Praktyczny algorytm implementujący odwrotna metodę potęgową.
\entry
Jeśli $(\lambda, v)$ - para własna macierzy A, to $(\lambda - \mu)$ - wartość własna macierzy $(A - \mu I)$
	\((A - \mu I)v = Av - \mu v = \lambda v - \mu v = (\lambda - \mu)v\)
\entry

% Najlepsze przybliżenie wartości własnej z użyciem ilorazu Rayleigh.
% (MP: 16 XI/1.5/23)
\entry
Znając przybliżony $x_k$ wek. wł.: mamy najlepsze (w sensie
średniokwadratowym) przybliżenie war. wł. (\textbf{iloraz Rayleigh}):
$\lambda_k \coloneqq x_k^T A x_k / x_k^T x_k$.
Gdy $x_k = v$:
$v^T A v / v^T v = v^T \lambda v / v^T v = \lambda$;

% Metoda Rayleigh (RQI).
% (MP: 16 XI/2/24)
\entry
\textbf{M. Rayleigh} (RQI):
$\text{for } k=0,\ldots$:
$x_{k+1} = (A - \mu_k I)^{-1} x_k$;
$x_{k+1} = x_{k+1} / \norm{x_{k+1}}$;
$\mu_{k+1} = x_{k+1}^T A x_{k+1}$.
Zbiega $\cdot^3$, a coraz gorsze uwarunkowanie ($A-\mu_k I$) pomaga.
Po $3$ iteracjach precyzja arytmetyki.

% TODO: Wartości własne macierzy blokowo-trójkątnej.
% (MP: 16 XI/2.5/3)

% Fakty o lokalizacji wartości własnych.
% (MP: 16 XI/3/4)
\entry
F. o lok. war. wł.:
$\abs{\lambda} \leq \norm{A}$;
% Definicja \sigma(A).
\entry
Zb. wartości wł. $A$:
$\sigma(A)$;

% Twierdzenie Gerszgorina.
\entry
Tw. Gerszgorina:
$\sigma(A) \in  \sum$ kół:
$K_i \coloneqq \set{z\in\mathbb{C}: \shortabs{z - a_{ii}} \leq \sum_{j \neq i}\shortabs{a_{ij}}}$;

% Twierdzenie (Bauer-Fike).
% (MP: 16 X/4/5)
\entry
Tw. (Bauer-Fike):
$A$ diagonalizowalna
($\exists_{X\text{ nieosobliwa}}$: $\Lambda \coloneqq X^{-1}AX=\mathrm{\lambda_i}_1^N$),
$\tilde{\lambda}$ war. wł. $\tilde{A} \coloneqq A + \Delta$.
Wtedy
$\min_{\lambda \in \sigma(A)} \shortabs{\lambda - \tilde{\lambda}} \leq
\cond[p]{X} \cdot \norm{\Delta}_p$;
% Wniosek z twierdzenia Bauera-Fike'a.
% (MP: 16 X/4.5/53)
% TODO: Dowód tego wniosku też jest ciekawy.
\entry
$A$ symetryczna:
$\min_i\shortabs{\lambda_i - \tilde{\lambda}_i} \leq \norm{\Delta}_2$;

% Metoda QR wyznaczania wszystkich wartości własnych macierzy (wraz z ulepszeniem i wariantem praktycznym).
\entry
M. QR na $\sigma(A)$:
$ A_1 \coloneqq A;
\fromloop[1]{k}\{
Q_k R_k \coloneqq A_k;
A_{k+1} \coloneqq R_k Q_k
\} $;
