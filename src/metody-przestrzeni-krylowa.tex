\mnsection{Metody przestrzeni Kryłowa}

% Definicja iteracji w metodach przestrzeni Kryłowa.
\entry
$k$-ta iteracja:
$x_k\in x_o+K_k$,
gdzie $K_k\coloneqq \set{r_0, Ar_0,\ldots, A^{k-1}r_0}$,
gdzie $r_0=b-Ax_0$;

% Metoda gradientów sprzężonych (CG).
\entry
\textbf{Metoda gradientów sprzężonych (CG)}:
$A=A^T>0$, $x_k\in x_0 + K_k$ t.,
że $\norm{x_k+x^*}_A\leq \norm{x-x^*}_A \forall x\in x_0 + K_k$,
gdzie $\norm{y}^2_A \coloneqq y^TAy$ ($\bigoh{N}$).
W ideal. arytmet. zbieżne do $x^*$ w $\leq N$ iter..
Po $k$ iter.:
$\norm{x_k - x^*}_A\leq 2(\frac{\sqrt{H} - 1}{\sqrt{H} + 1})^k\norm{x_0-x^*}_A$,
gdzie $H=\mathrm{cond}_2(A)$;

% TODO: Wzmianka o GMRES.
