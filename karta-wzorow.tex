\documentclass[10pt,a4paper,twocolumn]{article}

\usepackage{multicol}
\usepackage{amsbsy, amssymb, latexsym, amsmath, braket}
\usepackage[tiny]{titlesec}
\usepackage[hmargin=0.5cm,vmargin=1.0cm]{geometry}
\usepackage[utf8x]{inputenc}
\usepackage{polski}
\usepackage{scalefnt}
\usepackage[yyyymmdd,hhmmss]{datetime}
\usepackage{commath}
\usepackage{mathtools}
\usepackage{tikz}
\usepackage{xparse}
\usepackage{manfnt}
\usetikzlibrary{tikzmark}

% Definicje.
% https://tex.stackexchange.com/a/138901/80219
\newcommand{\Hsquare}{%
  \text{\fboxsep=-.2pt\fbox{\rule{0pt}{1ex}\rule{1ex}{0pt}}}%
}

% Hack, który sprawia, że \cdot w wyrażeniu \norm{A}\cdot\norm{A} znajduje się
% w równej odległości od obu norm. W przeciwnym wypadku \cdot znajduje się
% dziwnie blisko prawej normy.
% https://tex.stackexchange.com/a/99882/80219
% https://tex.stackexchange.com/a/61511/80219
\DeclareRobustCommand*{\norm}[1]{\|#1\|}

% Symbol "jak wyżej".
% https://tex.stackexchange.com/a/53844
\newcommand{\dittotikz}{%
    \tikz{
        % Pionowe kreski.
        \draw [line width=0.12ex] (-0.2ex,0) -- +(0,0.8ex)
            (0.2ex,0) -- +(0,0.8ex);
        % Poziome kreski.
        \draw [line width=0.08ex] (-0.6ex,0.4ex) -- +(-0.8em,0)
            (0.6ex,0.4ex) -- +(0.8em,0);
    }%
}

% Makra pomocnicze.
% Element najlepszej aproksymacji (ENA), zazwyczaj wyróżniany gwiazdką
% w górnym indeksie.
\newcommand{\ena}[1]{#1^{*}}
\newcommand{\enain}[2]{#1^{*} \in #2}
\newcommand{\forallin}[2]{\forall_{#1 \in #2}}
\newcommand{\existsin}[2]{\exists_{#1 \in #2}}
\newcommand{\maxin}[2]{\max_{#1 \in #2}}

% Krótkie \abs. Przydatne, gdy standardowe \abs wygląda niezgrabnie.
\newcommand{\shortabs}[1]{\mid\! #1\! \mid}

% Uwarunkowanie. Opcjonalny argument służy do podania normy.
% Na przykład: $\cond[2]{A}$.
\newcommand{\cond}[2][]{\mathrm{cond}_{#1}(#2)}

% Unormowanie wektora.
% Argumenty:
%   * wektor
%   * oznaczenie normy (opcjonalnie)
%   * potęga przy normie (opcjonalnie)
% Przykład: $\normify[\infty][2]{x_{k+1}}$
\DeclareDocumentCommand{\normify}{ O{} O{} m}{#3 = #3 / \norm{#3}_{#1}^{#2}}
\DeclareDocumentCommand{\normifyfrac}{ O{} O{} m}{#3 = \frac{#3}{ \norm{#3}_{#1}^{#2}}}

% Przydate do pętli iterujących nieznaną liczbę razy.
% Obowiązkowy argument to iterator, a opcjonalny argument to jego wartość
% początkowa domyślnie równa 0.
% Na przykład: $\fromloop[1]{k}$
\newcommand{\fromloop}[2][0]{\mathrm{from}(#2\!\!=\!\!#1)}

% Pochodna.
% Na przykład: $\deriv{w}{k}$
\newcommand{\deriv}[2]{#1^{(#2)}}

% Zbiór węzłów interpolacyjncyh.
% Indeksy po lewej: https://tex.stackexchange.com/a/64612/80219
% Na przykład: $\ipoints[i=0][n][a][b]{x_i}$
\DeclareDocumentCommand{\ipoints}{O{} O{} O{} O{} m}{\prescript{#4}{#3}{\{#5\}}^{#2}_{#1}}

% Ułamek bez kreski.
\DeclareDocumentCommand{\flatfrac}{m m}{#1 / #2}

% Notacja dużego O.
\DeclareDocumentCommand{\bigoh}{m}{\mathcal{O}(#1)}

\newcommand{\entry}{$\bullet$\hspace{0.15em}}
\newcommand{\subentry}{$\circledcirc$\hspace{0.15em}}
% https://tex.stackexchange.com/a/7045/80219
\newcommand{\textsubentry}[1]{\tikz[baseline=(char.base)]{
            \node[shape=circle,draw,inner sep=1pt] (char) {#1};\hspace{0.15em}}}

\newcommand{\mnsection}[1]{

    \hrulefill
    \textbf{{#1}}
    \hrulefill
}

% Wyłącz numerowanie stron.
\pagenumbering{gobble}

\setlength{\parindent}{0pt}
% Odległość pomiędzy liniami. Zmniejsz, jeżeli brakuje miejsca.
\setlength{\parskip}{0.5ex}

\title{Karta wzorów z metod numerycznych}

\begin{document}
% Rozmiar czcionki.
\scalefont{.8}

\text{\tiny{
    Wersja
    \input{|"git rev-list --count --first-parent HEAD"}z
    \today\ o \currenttime\ (\pdfmdfivesum file{./karta-wzorow.tex})
}}

\input{src/uklady-rownan-liniowych.tex}
\mnsection{Arytmetyka fl}

% TODO: Dokładniejszy opis.
\entry
Liczby maszynowe: $x=(-1)^s \cdot m \cdot \beta^e$, gdzie $0\leq m < \beta$ i $m=(f_0, \ldots, f_{p-1})_\beta$, $f_i\in\set{0,1,\ldots,\beta-1}$, gdzie $p$ to precyzja arytmetyki;

\entry
Liczby znormalizowane: $x=(-1)^s\cdot m \cdot 2^e$, gdzie $1\leq m < 2$;
\entry
Max błąd dla znormalizowanych: $ \frac{1}{2} \cdot \beta^{1 - m}$
\entry
$\mathrm{fl}(x) = x(1+\varepsilon_x)$, $\abs{\varepsilon_x}\leq \nu$
\entry
Błąd reprezentacji: $\shortabs{\mathrm{fl}(x) - x}$;
\entry
$\flatfrac{\shortabs{\mathrm{fl}(x) - x}}{\shortabs{x}}\leq 2^{-p}=\nu$, o ile nie ma *flow;
% Fused multiply-add.
\entry
$\mathrm{fl}(\mathrm{FMA}(a,b,c)) = \mathrm{fl}(a \cdot b + c)$;

\entry
Standard fl gwarantuje, że $a\, \Hsquare\, b = \mathrm{fl}(a \,\Hsquare\, b)$, jeżeli nie ma NaN i *flowów;

% Błędy w obliczeniach numerycznych.

\entry
\textbf{Błąd bezwzględny}:
$=\norm{\tilde{x} - x}$;
\entry
\textbf{Błąd względny}:
$=\frac{\text{błąd bezwzgl.}}{\norm{x}}$;

\entry
\textbf{Silna numeryczna poprawność} (NP):
alg. $A$ jest s. n. p.,
jeśli $\tilde{y} = P(\tilde{x})$,
gdzie $\norm{\tilde{x}-x} / x\leq k\cdot\nu$,
$P$---zadanie,
$\tilde{y}$---dokł. rozw. dla trochę zab. danych;

\entry
\textbf{Słaba numeryczna poprawność} (stabilność):
daje prawie dokł. rozw. dla prawie dokł. danych:
$\norm{\tilde{x}-x} / x\leq k\cdot\nu$, $\norm{\tilde{y}-P(\tilde{x})} / P(\tilde{x})\leq k\cdot\nu$;

\input{src/normy.tex}
\mnsection{Uwarunkowanie zadania}

\entry
Wskaźnik u. $P$ w punkcie $x$:
$\mathrm{cond}_{\mathrm{abs}}(P,x) \coloneqq \sup_{\text{małe }\delta} \frac{\norm{P(x+\delta) - P(x)}}{\norm{\delta}}$;

\entry
$\norm{P(x+\delta)} \leq \mathrm{cond}_{\mathrm{abs}}(P,x)\norm{\delta}$;
\entry
$\frac{\norm{P(x+\delta) - P(x)}}{\norm{P(x)}} \leq \mathrm{cond}_{\mathrm{rel}}(P,x)\frac{\norm{\delta}}{\norm{x}}$;

% Idealizacja.

\entry
$
\mathrm{cond}_{\mathrm{rel}}(P,x) \coloneqq
\mathrm{cond}_{\mathrm{abs}}(P,x) \frac{\norm{x}}{\norm{P(x)}}  \coloneqq
\lim\limits_{\norm{\delta \to 0}} \frac{\norm{P(x+\delta) - P(x)}}{\norm{\delta}} \cdot \frac{\norm{x}}{\norm{P(x)}}
$;

\entry
Zadanie $P$ jest źle uwarunkowane w punkcie $x$,
gdy $\mathrm{cond}(P,x) \gg 1$,
bo małe zaburzenie danych może spowodować duży błąd wyniku;

% XXX: Czy to powinny być normy drugie?
\entry
$\cond{A} \coloneqq \norm{\vphantom{A^{-1}}A}\cdot\norm{A^{-1}} > 0$;
\entry
$\cond{AB} \leq \cond{A}\cond{B}$;

\entry
$A=A^T\in\mathbb{R}^{N\times N}$,
to
$\mathrm{cond}_2(A)\coloneqq \frac{\max_i\abs{\lambda_i}}{\min_i\abs{\lambda_i}}$
i
$\exists_{v_i\neq 0} Av_i=\lambda_i v_i$;

\entry
$\exists_{v_i\neq 0} A_{v_i} = \lambda_i v_i$,
gdzie $v_i$ --- wektor własny;

% Wpływ zaburzania macierzy i prawej strony na uwarunkowanie.
% (MP: 19 X/3.5/172)
\entry
Gdy
$\varepsilon\cdot\mathrm{cond}(A) < \frac{1}{2}$,
to $\frac{\norm{x - \tilde{x}}}{\norm{x}} \leq 4 \cdot \mathrm{cond}(A) \cdot \varepsilon$,
gdzie $\varepsilon=\flatfrac{\norm{b-b'}}{\norm{b}}$;

% Przydatny wzór opisujący relację błędu względnego i zaburzeń danych.
% Pojawił się m. in. na kolokwium 2017/2018.
\entry
$\norm{x - \tilde{x}} / \norm{x} \leq \cond{A} \norm{b - b^{*}} / \norm{b}$;

\entry
Jeśli $\norm{B} < 1$,
to $I+B$ odwracalna i $\norm{(I+B)^{-1}} \leq 1 / (1 - \norm{B})$;

\entry
Jeśli $A$ odwracalna i $\norm{A^{-1}\Delta} < 1$,
to $(A+\Delta)^{-1}$ istnieje
i $\norm{(A+\Delta)^{-1}} \leq \frac{\norm{A^{-1}}}{1 - \norm{A^{-1}} \cdot \norm{\vphantom{A^1}\Delta}}$,
gdzie $\Delta \leq \varepsilon\norm{A}$;

% Numeryczne kryterium NP.
\entry
\textbf{Numeryczne kryterium NP}:
gdy $\tilde{x}$ przybliżonym rozw. $Ax=b$,\\
to $(A+\Delta)\tilde{x}=b+\delta$
i $\frac{\norm{\delta}}{\norm{b}}, \frac{\norm{\Delta}}{\norm{A}} \leq \varepsilon$,
gdzie $\varepsilon \coloneqq \frac{\norm{b-A\tilde{x}}}{\norm{A} \cdot \norm{\tilde{x}} + \norm{b} }$;

% TODO: Przykłady zadań dobrze i źle uwarunkowanych.

% Lemat.
\entry
$\norm{\Delta} < 1$,
to $I + \Delta$ nieosobliwa
oraz $\norm{(I + \Delta)^{-1}} \leq \frac{1}{1 - \norm{\Delta}}$;

% \input{src/macierze-rzadkie.tex}
\input{src/metody-stacjonarne.tex}
\mnsection{Metody przestrzeni Kryłowa}

% Definicja iteracji w metodach przestrzeni Kryłowa.
\entry
$k$-ta iteracja:
$x_k\in x_o+K_k$,
gdzie $K_k\coloneqq \set{r_0, Ar_0,\ldots, A^{k-1}r_0}$,
gdzie $r_0=b-Ax_0$;

% Metoda gradientów sprzężonych (CG).
\entry
\textbf{Metoda gradientów sprzężonych (CG)}:
$A=A^T>0$, $x_k\in x_0 + K_k$ t.,
że $\norm{x_k+x^*}_A\leq \norm{x-x^*}_A \forall x\in x_0 + K_k$,
gdzie $\norm{y}^2_A \coloneqq y^TAy$ ($\bigoh{N}$).
W ideal. arytmet. zbieżne do $x^*$ w $\leq N$ iter..
Po $k$ iter.:
$\norm{x_k - x^*}_A\leq 2(\frac{\sqrt{H} - 1}{\sqrt{H} + 1})^k\norm{x_0-x^*}_A$,
gdzie $H=\mathrm{cond}_2(A)$;

% TODO: Wzmianka o GMRES.

\input{src/lznk.tex}
\mnsection{Zadanie własne}

% Definicja zadania własnego.
\entry
Zadanie własne $A \in \mathbb{R}^{N \times N}$:
$(\lambda,x) \in \mathbb{C} \times \mathbb{C}^N$:
$Ax = \lambda x$;

% Fakt o wartościach własnych (1).
\entry
Gdy $\lambda$ dla $A$, to $(\lambda - \mu)$ dla $A-\mu I$;
% Fakt o wartościach własnych (2).
\entry
Gdy $\lambda$ dla $A$ nieos., to $\frac{1}{\lambda}$ dla $A^{-1}$;

% Metoda potęgowa wyznaczania wektora własnego.
% (9 XI/4.5/43)
\entry
\textbf{M}. \textbf{pot}.:
$\abs{\lambda_1} \!>\! \abs{\lambda_2} \!\geq\! \ldots \!\geq\! \abs{\lambda_n}
\!\!\Rightarrow\!
\fromloop{k}\{x_{k+1}=Ax_k; \normifyfrac[\infty][2]{x_{k+1}}\}$;

% Wniosek z powyższych faktów dla metody potęgowej.
\entry
M. pot. zastosowana do
$(A-\mu I)^{-1}$
będzie zbieżna do
$\lambda_i$,
czyli w. wł. najbliższej $\mu$.
Rzeczywiście, w. wł.
$(A-\mu I)^{-1}$
to
$\frac{1}{\lambda_i-\mu}$
i jeśli
$\frac{1}{\shortabs{\lambda_i-\mu}} > \frac{1}{\shortabs{\lambda_j-\mu}}, j\neq i$,
to
$\frac{1}{\lambda_i-\mu}$
jest dominująca;

% Odwrotna metoda potęgowa.
% (MP: 16 XI/1.5/2)
\entry
\textbf{Odw. m. pot.}:
$\text{for } k=0,\ldots$:
$x_{k+1} = (A-\mu I)^{-1}x_k$;
$x_{k+1} = x_{k+1} / \norm{x_{k+1}}$;
$\bigoh{n^3}$;
% Praktyczny algorytm implementujący odwrotna metodę potęgową.
\entry
Jeśli $(\lambda, v)$ - para własna macierzy A, to $(\lambda - \mu)$ - wartość własna macierzy $(A - \mu I)$
	\((A - \mu I)v = Av - \mu v = \lambda v - \mu v = (\lambda - \mu)v\)
\entry

% Najlepsze przybliżenie wartości własnej z użyciem ilorazu Rayleigh.
% (MP: 16 XI/1.5/23)
\entry
Znając przybliżony $x_k$ wek. wł.: mamy najlepsze (w sensie
średniokwadratowym) przybliżenie war. wł. (\textbf{iloraz Rayleigh}):
$\lambda_k \coloneqq x_k^T A x_k / x_k^T x_k$.
Gdy $x_k = v$:
$v^T A v / v^T v = v^T \lambda v / v^T v = \lambda$;

% Metoda Rayleigh (RQI).
% (MP: 16 XI/2/24)
\entry
\textbf{M. Rayleigh} (RQI):
$\text{for } k=0,\ldots$:
$x_{k+1} = (A - \mu_k I)^{-1} x_k$;
$x_{k+1} = x_{k+1} / \norm{x_{k+1}}$;
$\mu_{k+1} = x_{k+1}^T A x_{k+1}$.
Zbiega $\cdot^3$, a coraz gorsze uwarunkowanie ($A-\mu_k I$) pomaga.
Po $3$ iteracjach precyzja arytmetyki.

% TODO: Wartości własne macierzy blokowo-trójkątnej.
% (MP: 16 XI/2.5/3)

% Fakty o lokalizacji wartości własnych.
% (MP: 16 XI/3/4)
\entry
F. o lok. war. wł.:
$\abs{\lambda} \leq \norm{A}$;
% Definicja \sigma(A).
\entry
Zb. wartości wł. $A$:
$\sigma(A)$;

% Twierdzenie Gerszgorina.
\entry
Tw. Gerszgorina:
$\sigma(A) \in  \sum$ kół:
$K_i \coloneqq \set{z\in\mathbb{C}: \shortabs{z - a_{ii}} \leq \sum_{j \neq i}\shortabs{a_{ij}}}$;

% Twierdzenie (Bauer-Fike).
% (MP: 16 X/4/5)
\entry
Tw. (Bauer-Fike):
$A$ diagonalizowalna
($\exists_{X\text{ nieosobliwa}}$: $\Lambda \coloneqq X^{-1}AX=\mathrm{\lambda_i}_1^N$),
$\tilde{\lambda}$ war. wł. $\tilde{A} \coloneqq A + \Delta$.
Wtedy
$\min_{\lambda \in \sigma(A)} \shortabs{\lambda - \tilde{\lambda}} \leq
\cond[p]{X} \cdot \norm{\Delta}_p$;
% Wniosek z twierdzenia Bauera-Fike'a.
% (MP: 16 X/4.5/53)
% TODO: Dowód tego wniosku też jest ciekawy.
\entry
$A$ symetryczna:
$\min_i\shortabs{\lambda_i - \tilde{\lambda}_i} \leq \norm{\Delta}_2$;

% Metoda QR wyznaczania wszystkich wartości własnych macierzy (wraz z ulepszeniem i wariantem praktycznym).
\entry
M. QR na $\sigma(A)$:
$ A_1 \coloneqq A;
\fromloop[1]{k}\{
Q_k R_k \coloneqq A_k;
A_{k+1} \coloneqq R_k Q_k
\} $;
% Metoda QR z przesunięciem.
\entry
Lepiej:
$ A_1 \coloneqq A;
\fromloop[1]{k}\{
\text{wybierz przesunięcie } \sigma_k;
Q_k R_k \coloneqq A_k - \sigma_kI;
A_{k+1} \coloneqq R_k Q_k + \sigma_kI
\} $;
% Praktyczne QR.
\entry
Przed iteracją sprowadź $A$ do postaci Hassenberga lub trójdiagonalnej, gdy $A=A^T$;

\mnsection{Wielomiany}

% Interpolacja Lagrange'a.
\entry
\textbf{I. Lagrange'a}:
$w(x_i) = f(x_i)$;
% Interpolacja Hermite'a.
\entry
\textbf{I. Hermite'a}:
$\deriv{w}{k}(x_i) = \deriv{f}{k}_i, k = 0 \ldots m$;
% Twierdzenie o interpolacji.
\entry
Tw. o interpolacji:
$\ipoints[i=0][n][a=][=b]{x_i}, f(x_i)$,
to
$\exists!_{w \in \mathbb{P}_n} w(x_i)=f(x_i)$;
$w(x_j) = y_i = \sum_{i=0}^n\alpha_i\varphi_i(x_j) = y_i$;
% Baza naturalna i macierz Vandermonde'a.
\entry
Dla \textbf{bazy nat.}
$\varphi_j(x)=x^j$
mamy \textbf{m. Vandermonde'a}:
$[x^j_i]$
(gęsta, niesym., zwykle b. źle uwar.);

% Baza Newtona.
\entry
\textbf{Baza Newtona}:
$\varphi_i(x) \coloneqq (x-x_0)\cdots(x-x_i),
(w_n(x)-w_{n-1}(x) = b_n\varphi_n(x))$;
% Algorytm różnic dzielonych.
% (MP: 30 XI/3/1332)
\entry
\textbf{A. różnic dzielonych}:
$
f[x_0] = f(x_0),
f[x_0,\ldots,x_k] = (f[x_1,\ldots,x_k] - f[x_0,\ldots,x_{k-1}])/(x_k - x_0),
f[x_1, x_1, x_1] = \frac{f''(x_1)}{2!}
$;

% TODO: Zmodyfikowany algorytm Hornera dla wielomianów w bazie Newtona.
% (MP: 30 XI/4/134)
% (MP: 1 XII/1.5/3)

% Twierdzenie o błędzie interpolacji.
\entry
\textbf{Tw. o błędzie interpolacji}:
$f\in C^{n+1}[a;b], \ipoints[0][n][a=][=b]{x_i}, w_n(x)$ jest w. i. L.
Wtedy
$\forallin{x}{[a;b]} \existsin{\xi}{(a;b)} f(x) - w_n(x) = \frac{\deriv{f}{n+1}(\xi)}{(n+1)!}\omega_n(x)$,
gdzie $\omega_n(x)=\varphi_{n+1}(x)$;
% TODO: Interpolacja Hermite'a.
% (MP: 30 XI/4.5/15)
% Baza Lagrange'a.
\entry
\textbf{Baza Lagrange'a:}
$
l_i(x) = \frac{(x-x_0)\cdots(x-x_n)}{(x_i - x_0)\cdots(x_i - x_{i-1})(x_i - x_{i+1}) \cdots (x_i - x_n)},
l_i(x_j) = [i=j],
w(x_j)=f(x_j)\cdot 1
$;

% Algorytm barycentryczny dla wyznaczania wartości wielomianu interpolacyjnego Lagrange'a w bazie Lagrange'a.
\entry
A. barycentryczny dla w. i. L. w b. L.:
$\frac{w_i}{x - x_i}\varphi_{n+1}(x) \coloneqq l_i(x)$:
\textsubentry{1}
Wyznacz
$\{w_i\}$;
\textsubentry{2}
$w(x) = (\sum f(x_i)\frac{w_i}{x - x_i})\varphi_{n+1}(x)$;
$\mathcal{O}(n)$;

% Błąd interpolacji dla równoodległych węzłów.
% (MP: 1 XII/3/5)
\entry
\textbf{Błąd i. dla równoodległych węzłów:}
$f \in C^{n+1}[a;b]$, n+1 węzłów, $h = x_{i+1} - x_i$ to
$\max_{x \in [a;b]}\shortabs{f(x)-w(x)} \leq \frac{\max_\xi\shortabs{\deriv{f}{n+1}(\xi)} h^{n+1}}{4(n+1)}$,
$\shortabs{\varphi_{n+1}(x)} \leq \frac{1}{4} h^{n+1} n!$;

% Funkcja Rungego przykładem słabości interpolacji wielomianowej.
% (MP: 7 XII/1/1)
\entry
Interpolacja w. f. Rungego
$f(x)=1/(1+x^2)$
ma duży błąd.

\input{src/splajny.tex}
\input{src/aproksymacja-funkcji} % Wykład 21.12.
\input{src/aproksymacja-jednostajna.tex} % Wykład 11.01.
\mnsection{Równania nieliniowe}

% TODO: Twierdzenie Darboux.

% TODO: Twierdzenie o przedziałach lokalizujących zero w kolejnych krokach metody bisekcji.

\entry
Szybkość zbieżności:
wykładnicza$(p)$:
$\exists_{c \geq 0} \exists_{N > 0}
\abs{x_{n + 1} - x^{*}} = c \abs{x_n - x^{*}}^p \forall_{n \geq N}$;
liniowa$(\gamma)$:
$\exists_{\gamma \in [0;1)} \exists_{N > 0}
\abs{x_{n + 1} - x^{*}} = \gamma \abs{x_n - x^{*}} \forall_{n \geq N}$;

% TODO: Metoda Newtona: czy tyle wystarczy?
\entry
Metoda Newtona:
$f(x) = 0 = f(x+h) = f(x) + f'(x)h + (f''(x)h^2/2+\cdots$;
$x_{n+1} = x_n - f(x_n)/f'(x_n)$;

% Twierdzenie o zbieżności metody Newtona.
\entry
Tw. (zbieżność m. N.):
$f \in C^2[a;b], \exists_{\ena{x} \in (a;b)}$,
że $f(\ena{x})=0, f'(\ena{x}) \neq 0$.
Wtedy $\exists_{U \ni \ena{x}}$, że m. N. zbieżna do $\ena{x} \ \forall_{x_0 \in U}$.
Ponadto zbieżność kwadratowa:
$\exists_{c>0} \abs{x_{n+1} - \ena{x}} \leq c \abs{x_n - \ena{x}}^2 \forall_{n=0,\ldots}$;

% Własności metody Newtona.
\entry
Własności m. N.:
\subentry
$f$ musi być różniczkowalna;
\subentry
musi być znany wzór na pochodną;
\subentry
$f'(\ena{x}) \neq 0 \Rightarrow \ena{x}$ zerem jednokrotnym i gdy krotności $m$,
to metoda zbiega liniowo do
$\abs{x_{n+1} - \ena{x}} \approx (1-1/m) \cdot \abs{x_n - \ena{x}}$;

% Metoda Newtona dla 1/a.
\entry
M. N. dla
$\frac{1}{a}$: $x_{n+1} = x_n(2 - x_na)$, $0<x_0<\frac{2}{a}$;

% TODO: Metoda Herona.

% TODO: Metoda siecznych.

% Metoda Steffensena.
\entry
M. Steffensena:
$x_{n+1} = x_n - f(x_n) \cdot \frac{f(x_n)}{f(x_n + f(x_n)) - f(x_n)}, p=2$,
(problemy z niedomiarem);

% Metoda cięciw.
\entry
M. cięciw:
$x_{n+1} = x_n - f(x_n)/f'(x_0)$, (zb. liniowa);

% TODO: Metoda Halley'a.

% Metoda punktu stałego.
\entry
M. punktu stałego:
jak m. N. oraz
$x_n \rightarrow \ena{x} \Rightarrow \ena{x} = F(\ena{x})$;

% Twierdzenie Banacha o kontrakcji
\entry
\textbf{Tw. Banacha o kontrakcji}:
$F:[a;b] \Rightarrow [a;b]$
i spełnia warunki kontrakcji:
$\exists_{\gamma < 1} \forallin{x,y}{[a;b]} abs{x_{n+1} - \ena{x}} \leq \gamma \abs{x_n - \ena{x}}$
to kontrakcja ma jeden punkt stały $\ena{x}$
$x_{n+1} = F(x_n)$
jest zbieżny liniowo
$\forallin{x_o}{[a;b]}$
do $\ena{x}$
 % Wykład 18.01.
\mnsection{Kwadratura}

\entry
Kwadratura:
	\(
		Q(f) = \sum\limits_{i=0}^n a_if(x_i)
    \)
\entry
K. interpolacyjna:
    \(
        Q(f) = \int\limits_a^bw_f(x)dx
	\) dla $w_f$ - wiel. i. Lagrange'a
\entry
k. prostokątów:
	\(
		Q(f) = (b-a)\cdot f(\frac{a+b}{2})
	\)
\entry
k. trapezów:
	\(
		T(f) = \frac{b-a}{2}(f(a)+f(b))
	\)
\entry
k. parabol (Simpsona)
	\(
		P(f) = \frac{b-a}{6}(f(a) + 4f(\frac{a+b}{2})+f(b))
	\)
\entry
k. złożona
    \(
        \int\limits_a^b f(x)dx = \sum_{i=0}^{k-1} \int\limits_{x_i}^{x_{i+1}} f(x)dx
    \)
\entry
\textbf{Błąd kwadratur interpolacyjnych}
	Niech $f \in C^{n+1}(a, b)$ oraz $\norm{f^{n+1}}_{\infty, [a, b]} \leq M$ i niech $Q$ będzie kwadraturą interpolacyjną opartną na węzłąch $x_0, \cdots x_n \in [a, b]$. Wtedy:
	\(
		|Q(f) - I(f)| \leq \frac{M}{(n+1)!} \cdot |b-a|^{n+2}
	\)
\entry
błąd k. trapezów dla $f \in C^2[a, b]$:
	\(
		I(f) - \int\limits_a^bf(x)dx = \frac{(b-a)^3}{12}f^{(2)}(\xi_T)
	\), $\xi_T \in [a, b]$
\entry
błąd k. parabol dla $f \in C^4[a, b]$:
	\(
		P(f) - \int\limits_a^bf(x)dx = \frac{(b-a)^5}{2280}f^{(4)}(\xi_T)
	\), $\xi_T \in [a, b]$
\entry
textbf{błąd k. złożonych} dla $f \in C^{n}(a, b)$, k - liczba podziałów:
    \(
        |Q(f) - I(f)| \leq \frac{M}{(n+1)!} \cdot \frac{|b-a|^{n+2}}{k^{n+1}}
    \)
\entry
błąd k. Gaussa dla $f \in  C^{2n+2}[a, b]$, $Q$ - k.Gaussa oparta na $(n+1)$ węzłach:
	\(
		\int\limits_a^b f(x)\rho(x)dx - Q(f) = \norm{P_{n+1}}^2_\rho\cdot \frac{f^{(2n+2)}(\xi)}{(2n+2)!}
	\), $\xi \in [a, b]$, $P_{n+1}$ - $(n+1)$ wielomian ort. w $L^2_\rho(a, b)$
\entry
\textbf{Max rząd kwadratur i.:} $Q(f)$ - kwadratura i. na $(n+1)$ węzłach:
\(
	n+1 \leq \textrm{rząd}(Q) \leq 2(n+1) (=rząd k. Gaussa)
\)

\input{src/sztuczki.tex}

\end{document}
